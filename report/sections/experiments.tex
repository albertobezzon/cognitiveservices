\section{Experiments}

	\subsection{Experimental enviroments}
	
		Training a deep learning model that involves intensive compute tasks on large dataset can take days to run on a single CPU or a slow GPU. In our case, since HAM10000 dataset has 10015 images, it is unthinkable to perform the training of a convolutional neural network with a standard laptop. The solution turned to cloud computing. The choice fells on Google Cloud Platform because of the availability of free tier that consists in 300\$ free credits that can be used in any GCP product. 
		
		\smallskip
		
		We have tested our CNN models on a custom instance of Compute Engine. Our VM’s configuration is presented in Table \ref{tab:hw-config}
		
		\begin{table}[H]
			\small
			\begin{tabular}{ |>{\centering\arraybackslash}p{2.5cm}|c|c|c|c|>{\centering\arraybackslash}p{2.5cm}| }
				\hline
				\textbf{Operating System} & \textbf{CPU} & \textbf{Memory} & \textbf{Disk} & \textbf{GPU} & \textbf{Availability zone} \\ \hline
				
				Ubuntu 18.04 LTS & 8 core & 52 GB & SSD / 100 GB & 1x NVIDIA Tesla K80 &  europe-west1-b\\ \hline
				
			\end{tabular}
			\caption{Virtual machine configuration}
			\label{tab:hw-config}
		\end{table}
	
		Our models have been implemented in keras using tensorflow as a backend. The code is available on our GitHub repository: \url{https://github.com/albertobezzon/cognitiveservices}
		
		